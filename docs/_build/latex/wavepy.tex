%% Generated by Sphinx.
\def\sphinxdocclass{report}
\documentclass[letterpaper,10pt,english]{sphinxmanual}
\ifdefined\pdfpxdimen
   \let\sphinxpxdimen\pdfpxdimen\else\newdimen\sphinxpxdimen
\fi \sphinxpxdimen=.75bp\relax
\ifdefined\pdfimageresolution
    \pdfimageresolution= \numexpr \dimexpr1in\relax/\sphinxpxdimen\relax
\fi
%% let collapsible pdf bookmarks panel have high depth per default
\PassOptionsToPackage{bookmarksdepth=5}{hyperref}

\PassOptionsToPackage{booktabs}{sphinx}
\PassOptionsToPackage{colorrows}{sphinx}

\PassOptionsToPackage{warn}{textcomp}
\usepackage[utf8]{inputenc}
\ifdefined\DeclareUnicodeCharacter
% support both utf8 and utf8x syntaxes
  \ifdefined\DeclareUnicodeCharacterAsOptional
    \def\sphinxDUC#1{\DeclareUnicodeCharacter{"#1}}
  \else
    \let\sphinxDUC\DeclareUnicodeCharacter
  \fi
  \sphinxDUC{00A0}{\nobreakspace}
  \sphinxDUC{2500}{\sphinxunichar{2500}}
  \sphinxDUC{2502}{\sphinxunichar{2502}}
  \sphinxDUC{2514}{\sphinxunichar{2514}}
  \sphinxDUC{251C}{\sphinxunichar{251C}}
  \sphinxDUC{2572}{\textbackslash}
\fi
\usepackage{cmap}
\usepackage[T1]{fontenc}
\usepackage{amsmath,amssymb,amstext}
\usepackage{babel}



\usepackage{tgtermes}
\usepackage{tgheros}
\renewcommand{\ttdefault}{txtt}



\usepackage[Bjarne]{fncychap}
\usepackage{sphinx}

\fvset{fontsize=auto}
\usepackage{geometry}


% Include hyperref last.
\usepackage{hyperref}
% Fix anchor placement for figures with captions.
\usepackage{hypcap}% it must be loaded after hyperref.
% Set up styles of URL: it should be placed after hyperref.
\urlstyle{same}

\addto\captionsenglish{\renewcommand{\contentsname}{Contents:}}

\usepackage{sphinxmessages}
\setcounter{tocdepth}{2}



\title{WavePy}
\date{Aug 23, 2023}
\release{}
\author{Stefan Schaefers}
\newcommand{\sphinxlogo}{\vbox{}}
\renewcommand{\releasename}{}
\makeindex
\begin{document}

\ifdefined\shorthandoff
  \ifnum\catcode`\=\string=\active\shorthandoff{=}\fi
  \ifnum\catcode`\"=\active\shorthandoff{"}\fi
\fi

\pagestyle{empty}
\sphinxmaketitle
\pagestyle{plain}
\sphinxtableofcontents
\pagestyle{normal}
\phantomsection\label{\detokenize{index::doc}}


\sphinxAtStartPar
WavePy provides a simple python API to work with WAVE
\sphinxhref{https://www.helmholtz-berlin.de/forschung/oe/wi/optik-strahlrohre/arbeitsgebiete/ray\_en.html}{RAY\sphinxhyphen{}UI},
a software for undulators simulations developed at Helmholtz\sphinxhyphen{}Zentrum Berlin.

\sphinxAtStartPar
This do cuments contains the api descriptions of WavePy and some general
remarks.
WavePy is a python wrapper for WAVE that allows easy calculation of
spectra with wave from magnetic field data. Also functionality for the loading
and interp olation of magnetic field data is incorporated.

\sphinxstepscope


\chapter{Installation}
\label{\detokenize{installation:installation}}\label{\detokenize{installation::doc}}

\section{Install WavePy}
\label{\detokenize{installation:install-wavepy}}
\sphinxAtStartPar
Wavepy will works on Linux, macOS and Windows.
\begin{itemize}
\item {} 
\sphinxAtStartPar
You will need Python 3.9 or newer. From a shell (“Terminal” on OSX),
check your current Python version.

\begin{sphinxVerbatim}[commandchars=\\\{\}]
python3\PYG{+w}{ }\PYGZhy{}\PYGZhy{}version
\end{sphinxVerbatim}

\sphinxAtStartPar
If that version is less than 3.9, you must update it.

\sphinxAtStartPar
We recommend installing wavepy into a “virtual environment” so that this
installation will not interfere with any existing Python software:

\begin{sphinxVerbatim}[commandchars=\\\{\}]
python3\PYG{+w}{ }\PYGZhy{}m\PYG{+w}{ }venv\PYG{+w}{ }\PYGZti{}/wavepy\PYGZhy{}tutorial
\PYG{n+nb}{source}\PYG{+w}{ }\PYGZti{}/wavepy\PYGZhy{}tutorial/bin/activate
\end{sphinxVerbatim}

\sphinxAtStartPar
Alternatively, if you are a
\sphinxhref{https://conda.io/docs/user-guide/install/download.html}{conda} user,
you can create a conda environment:

\begin{sphinxVerbatim}[commandchars=\\\{\}]
conda\PYG{+w}{ }create\PYG{+w}{ }\PYGZhy{}n\PYG{+w}{ }wavepy\PYGZhy{}tutorial\PYG{+w}{ }\PYG{l+s+s2}{\PYGZdq{}python\PYGZgt{}=3.9\PYGZdq{}}
conda\PYG{+w}{ }activate\PYG{+w}{ }wavepy\PYGZhy{}tutorial
\end{sphinxVerbatim}

\begin{sphinxVerbatim}[commandchars=\\\{\}]
python3\PYG{+w}{ }\PYGZhy{}m\PYG{+w}{ }pip\PYG{+w}{ }install\PYG{+w}{ }\PYGZhy{}\PYGZhy{}upgrade\PYG{+w}{ }wavepy
\end{sphinxVerbatim}

\end{itemize}


\section{Troubleshooting}
\label{\detokenize{installation:troubleshooting}}

\subsection{Wave can not be Executed}
\label{\detokenize{installation:wave-can-not-be-executed}}
\sphinxAtStartPar
Make sure that \sphinxcode{\sphinxupquote{wavepy/WAVE/bin/wave.exe}} is executable.


\subsection{Files are not copied after the simulations runned}
\label{\detokenize{installation:files-are-not-copied-after-the-simulations-runned}}
\sphinxAtStartPar
Does the result folder exist? If not, create it.


\subsection{Interpolated files are not saved}
\label{\detokenize{installation:interpolated-files-are-not-saved}}
\sphinxAtStartPar
Does the folder where they should be saved exist? If not, create it.

\sphinxAtStartPar
WAVE is complaining about: NEGATIVE OR ZERO PHOTON EN\sphinxhyphen{}
ERGY OCCURED WHILE EXTENDING ENERGY
\textasciitilde{}\textasciitilde{}\textasciitilde{}\textasciitilde{}\textasciitilde{}\textasciitilde{}\textasciitilde{}\textasciitilde{}\textasciitilde{}\textasciitilde{}\textasciitilde{}\textasciitilde{}\textasciitilde{}\textasciitilde{}\textasciitilde{}\textasciitilde{}\textasciitilde{}\textasciitilde{}\textasciitilde{}\textasciitilde{}\textasciitilde{}\textasciitilde{}\textasciitilde{}\textasciitilde{}\textasciitilde{}\textasciitilde{}\textasciitilde{}\textasciitilde{}\textasciitilde{}\textasciitilde{}\textasciitilde{}\textasciitilde{}\textasciitilde{}\textasciitilde{}\textasciitilde{}\textasciitilde{}\textasciitilde{}\textasciitilde{}\textasciitilde{}\textasciitilde{}\textasciitilde{}\textasciitilde{}\textasciitilde{}\textasciitilde{}\textasciitilde{}\textasciitilde{}\textasciitilde{}\textasciitilde{}\textasciitilde{}\textasciitilde{}\textasciitilde{}\textasciitilde{}\textasciitilde{}\textasciitilde{}\textasciitilde{}\textasciitilde{}\textasciitilde{}\textasciitilde{}\textasciitilde{}\textasciitilde{}\textasciitilde{}\textasciitilde{}\textasciitilde{}\textasciitilde{}\textasciitilde{}\textasciitilde{}\textasciitilde{}\textasciitilde{}\textasciitilde{}\textasciitilde{}\textasciitilde{}\textasciitilde{}\textasciitilde{}\textasciitilde{}\textasciitilde{}\textasciitilde{}\textasciitilde{}\textasciitilde{}\textasciitilde{}\textasciitilde{}\textasciitilde{}\textasciitilde{}\textasciitilde{}
Increase the number of energy points \sphinxhyphen{} parameter \sphinxcode{\sphinxupquote{freq\_num}}.
Wave is extending the energy range you specified in order to calculate
the folding procedure and may, with too little points on which to
calculate, run into negative energies. This is especially important at
low energy values.


\subsection{Wave complains it cannot find a zip file while trying to plot}
\label{\detokenize{installation:wave-complains-it-cannot-find-a-zip-file-while-trying-to-plot}}
\sphinxAtStartPar
Were the results files not prop erly stored before?
Check if the data folder contains more than a zip file and
delete everything but the zip file.

\sphinxstepscope


\chapter{How To Use WavePy}
\label{\detokenize{tutorial:how-to-use-wavepy}}\label{\detokenize{tutorial::doc}}

\section{Axis Nomenclature}
\label{\detokenize{tutorial:axis-nomenclature}}
\sphinxAtStartPar
Flight direction of the electron is \sphinxcode{\sphinxupquote{x}} .
The vertical direction (in planer undulators) is called
\sphinxcode{\sphinxupquote{y}} and the horizontal is \sphinxcode{\sphinxupquote{z}}
.


\section{Spectrum Calculations}
\label{\detokenize{tutorial:spectrum-calculations}}
\sphinxAtStartPar
Import wavepy and create a wave instance

\begin{sphinxVerbatim}[commandchars=\\\{\}]
\PYG{k+kn}{from} \PYG{n+nn}{wavepy} \PYG{k+kn}{import} \PYG{n}{WaveFromB}
\PYG{n}{wave} \PYG{o}{=} \PYG{n}{WaveFromB}\PYG{p}{(}\PYG{l+s+s1}{\PYGZsq{}}\PYG{l+s+s1}{By}\PYG{l+s+s1}{\PYGZsq{}}\PYG{p}{)}
\end{sphinxVerbatim}

\sphinxAtStartPar
WAVE can perform many different tasks, one of which is calculating a
spectrum from a given B\sphinxhyphen{}field. This specific functionality is so far
covered in wavepy. You can calculate the spectra from a magnetic field
in Bydirection only by using \sphinxcode{\sphinxupquote{By}}, from By and Bz data
by setting \sphinxcode{\sphinxupquote{Byz}} and from all components by writing
\sphinxcode{\sphinxupquote{Bxyz}}. All field data should be in the format of two cols,
the first one containing \sphinxcode{\sphinxupquote{x}} in mm and the second B in Tesla,
no separators and header lines.
Next you can get standard wave parameters by writing:

\begin{sphinxVerbatim}[commandchars=\\\{\}]
\PYG{c+c1}{\PYGZsh{} Get Wave\PYGZhy{}parameter}
\PYG{n}{paras} \PYG{o}{=} \PYG{n}{wave}\PYG{o}{.}\PYG{n}{get\PYGZus{}paras}\PYG{p}{(}\PYG{p}{)}
\PYG{c+c1}{\PYGZsh{} Customize Parameters}
\PYG{n}{paras}\PYG{o}{.}\PYG{n}{field\PYGZus{}folder}\PYG{o}{.}\PYG{n}{set}\PYG{p}{(}\PYG{l+s+s1}{\PYGZsq{}}\PYG{l+s+s1}{..../Fields/Interpolated/}\PYG{l+s+s1}{\PYGZsq{}}\PYG{p}{)} \PYG{c+c1}{\PYGZsh{} Field Folder}
\PYG{n}{paras}\PYG{o}{.}\PYG{n}{field\PYGZus{}files}\PYG{o}{.}\PYG{n}{set}\PYG{p}{(} \PYG{p}{[} \PYG{l+s+s1}{\PYGZsq{}}\PYG{l+s+s1}{interp\PYGZus{}b\PYGZus{}field\PYGZus{}gap\PYGZus{}18.45\PYGZus{}.dat}\PYG{l+s+s1}{\PYGZsq{}}\PYG{p}{]} \PYG{p}{)} \PYG{c+c1}{\PYGZsh{} The magnetic field files to be used in the simulation}
\PYG{n}{paras}\PYG{o}{.}\PYG{n}{res\PYGZus{}folder}\PYG{o}{.}\PYG{n}{set}\PYG{p}{(}\PYG{l+s+s1}{\PYGZsq{}}\PYG{l+s+s1}{..../Spectrum\PYGZus{}Results/First\PYGZus{}Simu/}\PYG{l+s+s1}{\PYGZsq{}}\PYG{p}{)}   \PYG{c+c1}{\PYGZsh{} where to store the wave results}
\PYG{n}{paras}\PYG{o}{.}\PYG{n}{spec\PYGZus{}calc}\PYG{o}{.}\PYG{n}{set}\PYG{p}{(}\PYG{l+m+mi}{1}\PYG{p}{)}    \PYG{c+c1}{\PYGZsh{} calculate spectrum? if not, only trajectory is calculated}
\PYG{n}{paras}\PYG{o}{.}\PYG{n}{freq\PYGZus{}low}\PYG{o}{.}\PYG{n}{set}\PYG{p}{(}\PYG{l+m+mi}{100}\PYG{p}{)}   \PYG{c+c1}{\PYGZsh{} lower frequency for spectrum calc. [eV]}
\PYG{n}{paras}\PYG{o}{.}\PYG{n}{freq\PYGZus{}high}\PYG{o}{.}\PYG{n}{set}\PYG{p}{(}\PYG{l+m+mi}{105}\PYG{p}{)}  \PYG{c+c1}{\PYGZsh{} upper frequency for spectrum calc. [eV]}
\PYG{n}{paras}\PYG{o}{.}\PYG{n}{freq\PYGZus{}num}\PYG{o}{.}\PYG{n}{set}\PYG{p}{(}\PYG{l+m+mi}{5}\PYG{p}{)}     \PYG{c+c1}{\PYGZsh{} number of frequencies for spectrum calc. [eV]}
\PYG{n}{paras}\PYG{o}{.}\PYG{n}{pinh\PYGZus{}w}\PYG{o}{.}\PYG{n}{set}\PYG{p}{(}\PYG{l+m+mi}{3}\PYG{p}{)}       \PYG{c+c1}{\PYGZsh{} pinhole width (horizontal\PYGZhy{}z) [mm]}
\PYG{n}{paras}\PYG{o}{.}\PYG{n}{pinh\PYGZus{}h}\PYG{o}{.}\PYG{n}{set}\PYG{p}{(}\PYG{l+m+mi}{3}\PYG{p}{)}       \PYG{c+c1}{\PYGZsh{} pinhole height (vertical\PYGZhy{}y) [mm]}
\PYG{n}{paras}\PYG{o}{.}\PYG{n}{pinh\PYGZus{}x}\PYG{o}{.}\PYG{n}{set}\PYG{p}{(}\PYG{l+m+mi}{10}\PYG{p}{)}      \PYG{c+c1}{\PYGZsh{} pinhole distance in beam direction (x) [m]}
\PYG{n}{paras}\PYG{o}{.}\PYG{n}{pinh\PYGZus{}nz}\PYG{o}{.}\PYG{n}{set}\PYG{p}{(}\PYG{l+m+mi}{21}\PYG{p}{)}     \PYG{c+c1}{\PYGZsh{} number of horizontal points in pinhole for spec. calc.}
\PYG{n}{paras}\PYG{o}{.}\PYG{n}{pinh\PYGZus{}ny}\PYG{o}{.}\PYG{n}{set}\PYG{p}{(}\PYG{l+m+mi}{21}\PYG{p}{)}     \PYG{c+c1}{\PYGZsh{} number of vetical points in pinhole for spec. calc.}
\PYG{n}{paras}\PYG{o}{.}\PYG{n}{wave\PYGZus{}res\PYGZus{}copy\PYGZus{}behaviour}\PYG{o}{.}\PYG{n}{set}\PYG{p}{(}\PYG{l+s+s1}{\PYGZsq{}}\PYG{l+s+s1}{copy\PYGZus{}essentials}\PYG{l+s+s1}{\PYGZsq{}}\PYG{p}{)} \PYG{c+c1}{\PYGZsh{}\PYGZsq{}copy\PYGZus{}all\PYGZsq{}}
\end{sphinxVerbatim}

\sphinxAtStartPar
Paras is a \sphinxcode{\sphinxupquote{StdParamenter}} object. Then set the parameters and run wave:

\begin{sphinxVerbatim}[commandchars=\\\{\}]
\PYG{c+c1}{\PYGZsh{} Set Parameters}
\PYG{n}{wave}\PYG{o}{.}\PYG{n}{set\PYGZus{}paras}\PYG{p}{(}\PYG{n}{paras}\PYG{p}{)}
\PYG{c+c1}{\PYGZsh{} Run WAVE}
\PYG{n}{wave}\PYG{o}{.}\PYG{n}{run\PYGZus{}wave}\PYG{p}{(}\PYG{p}{)}
\end{sphinxVerbatim}


\subsection{Summary file}
\label{\detokenize{tutorial:summary-file}}
\sphinxAtStartPar
WavePy then saves all the data that is set to b e saved into
the designated folder and zips all data for storage.
On plotting the data is unzipped and zipped again.
Furthermore, a fiel \sphinxcode{\sphinxupquote{res\_summary.txt}} is stored
inside the zip archive. This file contains some general
information extracted from wave about the current simulation,
including:

\begin{sphinxVerbatim}[commandchars=\\\{\}]
\PYG{n}{bmax} \PYG{p}{[}\PYG{n}{T}\PYG{p}{]} \PYG{p}{:} \PYG{n+nb}{max}\PYG{o}{.} \PYG{n}{b} \PYG{n}{field}
\PYG{n}{bmin} \PYG{p}{[}\PYG{n}{T}\PYG{p}{]} \PYG{p}{:} \PYG{n+nb}{min}\PYG{o}{.} \PYG{n}{b} \PYG{n}{field}
\PYG{n}{first\PYGZus{}int} \PYG{p}{[}\PYG{n}{Tm}\PYG{p}{]} \PYG{p}{:} \PYG{n}{first} \PYG{n}{ingegral}
\PYG{n}{scnd\PYGZus{}int} \PYG{p}{[}\PYG{n}{Tmm}\PYG{p}{]} \PYG{p}{:} \PYG{n}{second} \PYG{n}{integral}
\PYG{n}{power} \PYG{p}{[}\PYG{n}{kW}\PYG{p}{]} \PYG{p}{:} \PYG{n}{total} \PYG{n}{emitted} \PYG{n}{power}
\PYG{n}{pinhole\PYGZus{}x} \PYG{p}{[}\PYG{n}{m}\PYG{p}{]} \PYG{p}{:} \PYG{n}{dist}\PYG{o}{.} \PYG{n}{of} \PYG{n}{pinhole} \PYG{k+kn}{from} \PYG{n+nn}{undulator} \PYG{n}{centre}
\PYG{n}{Fund}\PYG{o}{.} \PYG{n}{Freq}\PYG{o}{.} \PYG{p}{[}\PYG{n}{eV}\PYG{p}{]} \PYG{p}{:} \PYG{n}{fundamental} \PYG{n}{frequency}
\PYG{n}{Undu\PYGZus{}Para} \PYG{p}{:} \PYG{n}{undulator} \PYG{n}{parameter} \PYG{n}{K}
\PYG{n}{gamma} \PYG{p}{:} \PYG{n}{well}\PYG{p}{,} \PYG{n}{you} \PYG{n}{guessed} \PYG{n}{it}
\PYG{n}{half\PYGZus{}opening\PYGZus{}angle} \PYG{p}{[}\PYG{n}{rad}\PYG{p}{]} \PYG{p}{:} \PYG{n}{the} \PYG{n}{half}\PYG{o}{\PYGZhy{}}\PYG{n}{opening} \PYG{n}{angle}
\PYG{n}{cone\PYGZus{}radius\PYGZus{}at\PYGZus{}x} \PYG{p}{[}\PYG{n}{mm}\PYG{p}{]} \PYG{p}{:} \PYG{n}{the} \PYG{n}{cone}\PYG{o}{\PYGZhy{}}\PYG{n}{radius} \PYG{n}{at} \PYG{n}{the} \PYG{n}{position}
\PYG{n}{of} \PYG{n}{the} \PYG{n}{pinhole} \PYG{p}{(}\PYG{n}{tan}\PYG{p}{(}\PYG{n}{half\PYGZus{}opening\PYGZus{}angle}\PYG{p}{)}\PYG{o}{*}\PYG{n}{pinhole\PYGZus{}x}\PYG{p}{)}
\end{sphinxVerbatim}


\section{B\sphinxhyphen{}Field Interpolation}
\label{\detokenize{tutorial:b-field-interpolation}}
\sphinxAtStartPar
Keep in mind that WAVE exp ects the magnetic field data over mm.
You can load files containing magnetic field data for different gaps by
writing:

\begin{sphinxVerbatim}[commandchars=\\\{\}]
\PYG{n}{b\PYGZus{}field\PYGZus{}data} \PYG{o}{=} \PYG{n}{wpy}\PYG{o}{.}\PYG{n}{load\PYGZus{}b\PYGZus{}fields\PYGZus{}gap}\PYG{p}{(}\PYG{n}{folder}\PYG{p}{)}
\end{sphinxVerbatim}

\sphinxAtStartPar
The file name format should be \sphinxcode{\sphinxupquote{somename + gap\_x\_ + restname +.ending}}.
E.g. \sphinxcode{\sphinxupquote{myfile\_g\_23\_.dat}}. With the gap given in mm.
To plot the b\sphinxhyphen{}field data:

\begin{sphinxVerbatim}[commandchars=\\\{\}]
\PYG{n}{wpy}\PYG{o}{.}\PYG{n}{plot\PYGZus{}b\PYGZus{}field\PYGZus{}data}\PYG{p}{(} \PYG{n}{b\PYGZus{}fields} \PYG{o}{=} \PYG{n}{b\PYGZus{}field\PYGZus{}data} \PYG{p}{)}
\end{sphinxVerbatim}

\sphinxAtStartPar
You can center the data and cut all loaded magnetic fields to
the defined on the same interval by writing:

\begin{sphinxVerbatim}[commandchars=\\\{\}]
\PYG{n}{wpy}\PYG{o}{.}\PYG{n}{center\PYGZus{}b\PYGZus{}field\PYGZus{}data}\PYG{p}{(} \PYG{n}{b\PYGZus{}fields} \PYG{o}{=} \PYG{n}{b\PYGZus{}field\PYGZus{}data}\PYG{p}{,} \PYG{n}{lim\PYGZus{}peak} \PYG{o}{=} \PYG{l+m+mf}{0.01} \PYG{p}{)}
\PYG{n}{wpy}\PYG{o}{.}\PYG{n}{cut\PYGZus{}data\PYGZus{}support}\PYG{p}{(}\PYG{n}{b\PYGZus{}fields} \PYG{o}{=} \PYG{n}{b\PYGZus{}field\PYGZus{}data}\PYG{p}{,} \PYG{n}{col\PYGZus{}cut} \PYG{o}{=} \PYG{l+s+s1}{\PYGZsq{}}\PYG{l+s+s1}{x}\PYG{l+s+s1}{\PYGZsq{}}\PYG{p}{)}
\end{sphinxVerbatim}

\sphinxAtStartPar
The centering is done by identifying the first and last
peaks and centering those. The option \sphinxcode{\sphinxupquote{lim\_peak}}
gives the minimal height a p eak has to have to be identified as a
peak \sphinxhyphen{} this is necessary to account for noise in the data.
Save the pro cessed fields by writing:

\begin{sphinxVerbatim}[commandchars=\\\{\}]
\PYG{n}{wpy}\PYG{o}{.}\PYG{n}{save\PYGZus{}prepared\PYGZus{}b\PYGZus{}data}\PYG{p}{(}\PYG{n}{b\PYGZus{}fields} \PYG{o}{=} \PYG{n}{b\PYGZus{}field\PYGZus{}data}\PYG{p}{,} \PYG{n}{folder} \PYG{o}{=} \PYG{n}{folder}\PYG{p}{)}
\end{sphinxVerbatim}

\sphinxAtStartPar
Next we can interpolate the data by writing:

\begin{sphinxVerbatim}[commandchars=\\\{\}]
\PYG{n}{interp\PYGZus{}field} \PYG{o}{=} \PYG{n}{wpy}\PYG{o}{.}\PYG{n}{interpolate\PYGZus{}b\PYGZus{}data}\PYG{p}{(}\PYG{n}{b\PYGZus{}fields} \PYG{o}{=} \PYG{n}{b\PYGZus{}field\PYGZus{}data}\PYG{p}{,} \PYG{n}{gap} \PYG{o}{=} \PYG{n}{gap}\PYG{p}{,} \PYG{n}{lim\PYGZus{}peak} \PYG{o}{=} \PYG{l+m+mf}{0.01}\PYG{p}{)}
\end{sphinxVerbatim}

\sphinxAtStartPar
Where \sphinxcode{\sphinxupquote{gap\textasciigrave{}}} is the gap at which you’d like to interpolate.


\section{Troubleshooting}
\label{\detokenize{tutorial:troubleshooting}}

\subsection{Files are not copied after the simulations runned}
\label{\detokenize{tutorial:files-are-not-copied-after-the-simulations-runned}}
\sphinxAtStartPar
Does the result folder exist? If not, create it.


\subsection{Interpolated files are not saved}
\label{\detokenize{tutorial:interpolated-files-are-not-saved}}
\sphinxAtStartPar
Does the folder where they should be saved exist? If not, create it.

\sphinxAtStartPar
WAVE is complaining about: NEGATIVE OR ZERO PHOTON EN\sphinxhyphen{}
ERGY OCCURED WHILE EXTENDING ENERGY
\textasciitilde{}\textasciitilde{}\textasciitilde{}\textasciitilde{}\textasciitilde{}\textasciitilde{}\textasciitilde{}\textasciitilde{}\textasciitilde{}\textasciitilde{}\textasciitilde{}\textasciitilde{}\textasciitilde{}\textasciitilde{}\textasciitilde{}\textasciitilde{}\textasciitilde{}\textasciitilde{}\textasciitilde{}\textasciitilde{}\textasciitilde{}\textasciitilde{}\textasciitilde{}\textasciitilde{}\textasciitilde{}\textasciitilde{}\textasciitilde{}\textasciitilde{}\textasciitilde{}\textasciitilde{}\textasciitilde{}\textasciitilde{}\textasciitilde{}\textasciitilde{}\textasciitilde{}\textasciitilde{}\textasciitilde{}\textasciitilde{}\textasciitilde{}\textasciitilde{}\textasciitilde{}\textasciitilde{}\textasciitilde{}\textasciitilde{}\textasciitilde{}\textasciitilde{}\textasciitilde{}\textasciitilde{}\textasciitilde{}\textasciitilde{}\textasciitilde{}\textasciitilde{}\textasciitilde{}\textasciitilde{}\textasciitilde{}\textasciitilde{}\textasciitilde{}\textasciitilde{}\textasciitilde{}\textasciitilde{}\textasciitilde{}\textasciitilde{}\textasciitilde{}\textasciitilde{}\textasciitilde{}\textasciitilde{}\textasciitilde{}\textasciitilde{}\textasciitilde{}\textasciitilde{}\textasciitilde{}\textasciitilde{}\textasciitilde{}\textasciitilde{}\textasciitilde{}\textasciitilde{}\textasciitilde{}\textasciitilde{}\textasciitilde{}\textasciitilde{}\textasciitilde{}\textasciitilde{}\textasciitilde{}
Increase the number of energy points \sphinxhyphen{} parameter \sphinxcode{\sphinxupquote{freq\_num}}.
Wave is extending the energy range you specified in order to calculate
the folding procedure and may, with too little points on which to
calculate, run into negative energies. This is especially important at
low energy values.


\subsection{Wave complains it cannot find a zip file while trying to plot}
\label{\detokenize{tutorial:wave-complains-it-cannot-find-a-zip-file-while-trying-to-plot}}
\sphinxAtStartPar
Were the results files not prop erly stored before?
Check if the data folder contains more than a zip file and
delete everything but the zip file.

\sphinxstepscope


\chapter{How To Guides}
\label{\detokenize{how_to:how-to-guides}}\label{\detokenize{how_to::doc}}
\sphinxAtStartPar
nothing here at the moment

\sphinxstepscope


\chapter{API}
\label{\detokenize{API:api}}\label{\detokenize{API::doc}}

\section{Simulation}
\label{\detokenize{API:simulation}}

\subsection{WAVE}
\label{\detokenize{API:wave}}\index{WaveFromB (class in wavepy.wave)@\spxentry{WaveFromB}\spxextra{class in wavepy.wave}}

\begin{fulllineitems}
\phantomsection\label{\detokenize{API:wavepy.wave.WaveFromB}}
\pysigstartsignatures
\pysiglinewithargsret{\sphinxbfcode{\sphinxupquote{class\DUrole{w,w}{  }}}\sphinxcode{\sphinxupquote{wavepy.wave.}}\sphinxbfcode{\sphinxupquote{WaveFromB}}}{\sphinxparam{\DUrole{n,n}{b\_type}\DUrole{o,o}{=}\DUrole{default_value}{\textquotesingle{}By\textquotesingle{}}}}{}
\pysigstopsignatures
\sphinxAtStartPar
WAVE class to perform spectrum calculations from b\sphinxhyphen{}field data files

\sphinxAtStartPar
Initiates WAVE class object with self.std WAVE parameters
b\_type sets which b\sphinxhyphen{}fields are loaded from file: ‘y’: By,
‘yz’ By and Bz, ‘xyz’ … . Each B\sphinxhyphen{}field from different file
\begin{quote}\begin{description}
\sphinxlineitem{Parameters}
\sphinxAtStartPar
\sphinxstyleliteralstrong{\sphinxupquote{b\_type}} (\sphinxstyleliteralemphasis{\sphinxupquote{str}}\sphinxstyleliteralemphasis{\sphinxupquote{, }}\sphinxstyleliteralemphasis{\sphinxupquote{optional}}) \textendash{} Field. possible values: By, Byz, Bxyz. Defaults to ‘By’.

\end{description}\end{quote}
\index{get\_paras() (wavepy.wave.WaveFromB method)@\spxentry{get\_paras()}\spxextra{wavepy.wave.WaveFromB method}}

\begin{fulllineitems}
\phantomsection\label{\detokenize{API:wavepy.wave.WaveFromB.get_paras}}
\pysigstartsignatures
\pysiglinewithargsret{\sphinxbfcode{\sphinxupquote{get\_paras}}}{}{}
\pysigstopsignatures
\sphinxAtStartPar
Returns current parameters

\end{fulllineitems}

\index{run\_wave() (wavepy.wave.WaveFromB method)@\spxentry{run\_wave()}\spxextra{wavepy.wave.WaveFromB method}}

\begin{fulllineitems}
\phantomsection\label{\detokenize{API:wavepy.wave.WaveFromB.run_wave}}
\pysigstartsignatures
\pysiglinewithargsret{\sphinxbfcode{\sphinxupquote{run\_wave}}}{}{}
\pysigstopsignatures
\sphinxAtStartPar
Runs wave and postprocess the results

\end{fulllineitems}

\index{set\_paras() (wavepy.wave.WaveFromB method)@\spxentry{set\_paras()}\spxextra{wavepy.wave.WaveFromB method}}

\begin{fulllineitems}
\phantomsection\label{\detokenize{API:wavepy.wave.WaveFromB.set_paras}}
\pysigstartsignatures
\pysiglinewithargsret{\sphinxbfcode{\sphinxupquote{set\_paras}}}{\sphinxparam{\DUrole{n,n}{wave\_paras}\DUrole{p,p}{:}\DUrole{w,w}{  }\DUrole{n,n}{{\hyperref[\detokenize{API:wavepy.standard_parameters.StdParameters}]{\sphinxcrossref{StdParameters}}}}}}{}
\pysigstopsignatures
\sphinxAtStartPar
Sets new parameters
\begin{quote}\begin{description}
\sphinxlineitem{Parameters}
\sphinxAtStartPar
\sphinxstyleliteralstrong{\sphinxupquote{wave\_paras}} ({\hyperref[\detokenize{API:wavepy.standard_parameters.StdParameters}]{\sphinxcrossref{\sphinxstyleliteralemphasis{\sphinxupquote{StdParameters}}}}}) \textendash{} Standard Parameters

\end{description}\end{quote}

\end{fulllineitems}


\end{fulllineitems}



\subsection{WAVE API}
\label{\detokenize{API:wave-api}}\index{WaveAPI (class in wavepy.wave\_API)@\spxentry{WaveAPI}\spxextra{class in wavepy.wave\_API}}

\begin{fulllineitems}
\phantomsection\label{\detokenize{API:wavepy.wave_API.WaveAPI}}
\pysigstartsignatures
\pysiglinewithargsret{\sphinxbfcode{\sphinxupquote{class\DUrole{w,w}{  }}}\sphinxcode{\sphinxupquote{wavepy.wave\_API.}}\sphinxbfcode{\sphinxupquote{WaveAPI}}}{\sphinxparam{\DUrole{n,n}{wave\_folder}\DUrole{p,p}{:}\DUrole{w,w}{  }\DUrole{n,n}{str}}, \sphinxparam{\DUrole{n,n}{current\_folder}\DUrole{p,p}{:}\DUrole{w,w}{  }\DUrole{n,n}{str}\DUrole{w,w}{  }\DUrole{o,o}{=}\DUrole{w,w}{  }\DUrole{default_value}{False}}}{}
\pysigstopsignatures
\sphinxAtStartPar
API for the WAVE program
\begin{quote}\begin{description}
\sphinxlineitem{Parameters}\begin{itemize}
\item {} 
\sphinxAtStartPar
\sphinxstyleliteralstrong{\sphinxupquote{current\_folder}} (\sphinxstyleliteralemphasis{\sphinxupquote{str}}) \textendash{} \_description\_

\item {} 
\sphinxAtStartPar
\sphinxstyleliteralstrong{\sphinxupquote{wave\_folder}} (\sphinxstyleliteralemphasis{\sphinxupquote{str}}) \textendash{} \_description\_

\end{itemize}

\end{description}\end{quote}
\index{run() (wavepy.wave\_API.WaveAPI method)@\spxentry{run()}\spxextra{wavepy.wave\_API.WaveAPI method}}

\begin{fulllineitems}
\phantomsection\label{\detokenize{API:wavepy.wave_API.WaveAPI.run}}
\pysigstartsignatures
\pysiglinewithargsret{\sphinxbfcode{\sphinxupquote{run}}}{}{}
\pysigstopsignatures
\sphinxAtStartPar
Run Wave from the self.wave\_folder.

\sphinxAtStartPar
If given, change the directory back to self.current\_folder

\end{fulllineitems}


\end{fulllineitems}



\subsection{Elements}
\label{\detokenize{API:elements}}\index{WaveAttribute (class in wavepy.wave\_elements)@\spxentry{WaveAttribute}\spxextra{class in wavepy.wave\_elements}}

\begin{fulllineitems}
\phantomsection\label{\detokenize{API:wavepy.wave_elements.WaveAttribute}}
\pysigstartsignatures
\pysiglinewithargsret{\sphinxbfcode{\sphinxupquote{class\DUrole{w,w}{  }}}\sphinxcode{\sphinxupquote{wavepy.wave\_elements.}}\sphinxbfcode{\sphinxupquote{WaveAttribute}}}{\sphinxparam{\DUrole{n,n}{value}\DUrole{o,o}{=}\DUrole{default_value}{None}}, \sphinxparam{\DUrole{n,n}{name}\DUrole{o,o}{=}\DUrole{default_value}{None}}}{}
\pysigstopsignatures
\sphinxAtStartPar
Represents a wave attribute with a value.
\begin{quote}\begin{description}
\sphinxlineitem{Parameters}\begin{itemize}
\item {} 
\sphinxAtStartPar
\sphinxstyleliteralstrong{\sphinxupquote{value}} \textendash{} Initial value of the attribute.

\item {} 
\sphinxAtStartPar
\sphinxstyleliteralstrong{\sphinxupquote{name}} (\sphinxstyleliteralemphasis{\sphinxupquote{str}}\sphinxstyleliteralemphasis{\sphinxupquote{, }}\sphinxstyleliteralemphasis{\sphinxupquote{optional}}) \textendash{} Name of the attribute.

\end{itemize}

\end{description}\end{quote}

\end{fulllineitems}

\index{WaveElement (class in wavepy.wave\_elements)@\spxentry{WaveElement}\spxextra{class in wavepy.wave\_elements}}

\begin{fulllineitems}
\phantomsection\label{\detokenize{API:wavepy.wave_elements.WaveElement}}
\pysigstartsignatures
\pysiglinewithargsret{\sphinxbfcode{\sphinxupquote{class\DUrole{w,w}{  }}}\sphinxcode{\sphinxupquote{wavepy.wave\_elements.}}\sphinxbfcode{\sphinxupquote{WaveElement}}}{\sphinxparam{\DUrole{n,n}{b\_type}\DUrole{o,o}{=}\DUrole{default_value}{\textquotesingle{}By\textquotesingle{}}}}{}
\pysigstopsignatures
\sphinxAtStartPar
Represents a wave element with children.
\begin{quote}\begin{description}
\sphinxlineitem{Parameters}
\sphinxAtStartPar
\sphinxstyleliteralstrong{\sphinxupquote{b\_type}} (\sphinxstyleliteralemphasis{\sphinxupquote{str}}\sphinxstyleliteralemphasis{\sphinxupquote{, }}\sphinxstyleliteralemphasis{\sphinxupquote{optional}}) \textendash{} Type of the wave element.

\end{description}\end{quote}
\index{children() (wavepy.wave\_elements.WaveElement method)@\spxentry{children()}\spxextra{wavepy.wave\_elements.WaveElement method}}

\begin{fulllineitems}
\phantomsection\label{\detokenize{API:wavepy.wave_elements.WaveElement.children}}
\pysigstartsignatures
\pysiglinewithargsret{\sphinxbfcode{\sphinxupquote{children}}}{}{}
\pysigstopsignatures
\sphinxAtStartPar
Yields the children of the WaveElement.

\end{fulllineitems}


\end{fulllineitems}

\index{StdParameters (class in wavepy.standard\_parameters)@\spxentry{StdParameters}\spxextra{class in wavepy.standard\_parameters}}

\begin{fulllineitems}
\phantomsection\label{\detokenize{API:wavepy.standard_parameters.StdParameters}}
\pysigstartsignatures
\pysiglinewithargsret{\sphinxbfcode{\sphinxupquote{class\DUrole{w,w}{  }}}\sphinxcode{\sphinxupquote{wavepy.standard\_parameters.}}\sphinxbfcode{\sphinxupquote{StdParameters}}}{\sphinxparam{\DUrole{n,n}{b\_type}\DUrole{o,o}{=}\DUrole{default_value}{\textquotesingle{}By\textquotesingle{}}}}{}
\pysigstopsignatures
\sphinxAtStartPar
Represents standard parameters for wave simulations.
\begin{quote}\begin{description}
\sphinxlineitem{Parameters}
\sphinxAtStartPar
\sphinxstyleliteralstrong{\sphinxupquote{b\_type}} (\sphinxstyleliteralemphasis{\sphinxupquote{str}}\sphinxstyleliteralemphasis{\sphinxupquote{, }}\sphinxstyleliteralemphasis{\sphinxupquote{optional}}) \textendash{} Type of the wave element.

\end{description}\end{quote}
\index{get\_std\_paras() (wavepy.standard\_parameters.StdParameters method)@\spxentry{get\_std\_paras()}\spxextra{wavepy.standard\_parameters.StdParameters method}}

\begin{fulllineitems}
\phantomsection\label{\detokenize{API:wavepy.standard_parameters.StdParameters.get_std_paras}}
\pysigstartsignatures
\pysiglinewithargsret{\sphinxbfcode{\sphinxupquote{get\_std\_paras}}}{\sphinxparam{\DUrole{n,n}{b\_type}\DUrole{o,o}{=}\DUrole{default_value}{\textquotesingle{}By\textquotesingle{}}}}{}
\pysigstopsignatures
\sphinxAtStartPar
Returns a object containing WAVE standard parameters updated with standard values
\begin{quote}\begin{description}
\sphinxlineitem{Parameters}\begin{itemize}
\item {} 
\sphinxAtStartPar
\sphinxstyleliteralstrong{\sphinxupquote{b\_type}} (\sphinxstyleliteralemphasis{\sphinxupquote{str}}) \textendash{} Type of b\sphinxhyphen{}calculation. Options: ‘By’ (only By given), ‘Byz’, ‘Bxyz’ (each b\sphinxhyphen{}field in different file).

\item {} 
\sphinxAtStartPar
\sphinxstyleliteralstrong{\sphinxupquote{wave\_prog\_folder}} (\sphinxstyleliteralemphasis{\sphinxupquote{str}}) \textendash{} Main folder where wave is stored (ending on ‘/’).

\item {} 
\sphinxAtStartPar
\sphinxstyleliteralstrong{\sphinxupquote{in\_file\_folder}} (\sphinxstyleliteralemphasis{\sphinxupquote{str}}) \textendash{} Folder in which the wave in\sphinxhyphen{}files are stored (ending on ‘/’).

\item {} 
\sphinxAtStartPar
\sphinxstyleliteralstrong{\sphinxupquote{in\_files}} (\sphinxstyleliteralemphasis{\sphinxupquote{dict}}) \textendash{} Dictionary of wave in\sphinxhyphen{}files for different b\_type situations. Format: \{b\_type: wave\_in\_file\}.

\item {} 
\sphinxAtStartPar
\sphinxstyleliteralstrong{\sphinxupquote{field\_folder}} (\sphinxstyleliteralemphasis{\sphinxupquote{str}}) \textendash{} Folder where b\sphinxhyphen{}field files are stored.

\item {} 
\sphinxAtStartPar
\sphinxstyleliteralstrong{\sphinxupquote{field\_files}} (\sphinxstyleliteralemphasis{\sphinxupquote{list}}) \textendash{} List of b\sphinxhyphen{}field files. Format: 2 cols \sphinxhyphen{} x{[}mm{]} and B{[}T{]}, no separator, no headers.

\item {} 
\sphinxAtStartPar
\sphinxstyleliteralstrong{\sphinxupquote{res\_folder}} (\sphinxstyleliteralemphasis{\sphinxupquote{str}}) \textendash{} Folder where results are stored (ending on ‘/’).

\item {} 
\sphinxAtStartPar
\sphinxstyleliteralstrong{\sphinxupquote{wave\_data\_res\_folder}} (\sphinxstyleliteralemphasis{\sphinxupquote{str}}) \textendash{} Subfolder of res\_folder where wave data is stored.

\item {} 
\sphinxAtStartPar
\sphinxstyleliteralstrong{\sphinxupquote{pics\_folder}} (\sphinxstyleliteralemphasis{\sphinxupquote{str}}) \textendash{} Subfolder of res\_folder where pictures are stored.

\item {} 
\sphinxAtStartPar
\sphinxstyleliteralstrong{\sphinxupquote{res\_summary\_file}} (\sphinxstyleliteralemphasis{\sphinxupquote{str}}) \textendash{} Name of the summary file to be written.

\item {} 
\sphinxAtStartPar
\sphinxstyleliteralstrong{\sphinxupquote{no\_copy}} (\sphinxstyleliteralemphasis{\sphinxupquote{list}}) \textendash{} List of file names not to be copied/moved after simulation from wave stage folder.

\item {} 
\sphinxAtStartPar
\sphinxstyleliteralstrong{\sphinxupquote{wave\_ending\_extract}} (\sphinxstyleliteralemphasis{\sphinxupquote{list}}) \textendash{} List of file endings to move from wave stage folder after simulation.

\item {} 
\sphinxAtStartPar
\sphinxstyleliteralstrong{\sphinxupquote{wave\_ending\_copy}} (\sphinxstyleliteralemphasis{\sphinxupquote{list}}) \textendash{} List of file endings of files to be copied (not moved).

\item {} 
\sphinxAtStartPar
\sphinxstyleliteralstrong{\sphinxupquote{wave\_res\_copy\_behaviour}} (\sphinxstyleliteralemphasis{\sphinxupquote{str}}) \textendash{} Behavior for copying wave results: ‘copy\_all’, ‘copy\_del\_none’, ‘copy\_essentials’.

\item {} 
\sphinxAtStartPar
\sphinxstyleliteralstrong{\sphinxupquote{wave\_files\_essentials}} (\sphinxstyleliteralemphasis{\sphinxupquote{list}}) \textendash{} List of essential files when wave\_res\_copy\_behaviour is set to ‘copy\_essentials’.

\item {} 
\sphinxAtStartPar
\sphinxstyleliteralstrong{\sphinxupquote{zip\_res\_folder}} (\sphinxstyleliteralemphasis{\sphinxupquote{bool}}) \textendash{} Truth value, whether to zip results or not.

\item {} 
\sphinxAtStartPar
\sphinxstyleliteralstrong{\sphinxupquote{freq\_low}} (\sphinxstyleliteralemphasis{\sphinxupquote{float}}) \textendash{} Lower frequency (energy) of spectrum to calculate {[}eV{]}.

\item {} 
\sphinxAtStartPar
\sphinxstyleliteralstrong{\sphinxupquote{freq\_high}} (\sphinxstyleliteralemphasis{\sphinxupquote{float}}) \textendash{} Upper frequency (energy) of spectrum to calculate {[}eV{]}.

\item {} 
\sphinxAtStartPar
\sphinxstyleliteralstrong{\sphinxupquote{freq\_num}} (\sphinxstyleliteralemphasis{\sphinxupquote{int}}) \textendash{} Number of frequencies to calculate.

\item {} 
\sphinxAtStartPar
\sphinxstyleliteralstrong{\sphinxupquote{beam\_en}} (\sphinxstyleliteralemphasis{\sphinxupquote{float}}) \textendash{} Beam energy in {[}GeV{]}.

\item {} 
\sphinxAtStartPar
\sphinxstyleliteralstrong{\sphinxupquote{current}} (\sphinxstyleliteralemphasis{\sphinxupquote{float}}) \textendash{} Current in {[}A{]}.

\item {} 
\sphinxAtStartPar
\sphinxstyleliteralstrong{\sphinxupquote{pinh\_w}} (\sphinxstyleliteralemphasis{\sphinxupquote{float}}) \textendash{} Pinhole width (horizontal\sphinxhyphen{}z) {[}mm{]}.

\item {} 
\sphinxAtStartPar
\sphinxstyleliteralstrong{\sphinxupquote{pinh\_h}} (\sphinxstyleliteralemphasis{\sphinxupquote{float}}) \textendash{} Pinhole height (vertical\sphinxhyphen{}y) {[}mm{]}.

\item {} 
\sphinxAtStartPar
\sphinxstyleliteralstrong{\sphinxupquote{spec\_calc}} (\sphinxstyleliteralemphasis{\sphinxupquote{bool}}) \textendash{} Truth value, whether to calculate spectrum or only write trajectory.

\item {} 
\sphinxAtStartPar
\sphinxstyleliteralstrong{\sphinxupquote{pinh\_x}} (\sphinxstyleliteralemphasis{\sphinxupquote{float}}) \textendash{} Position of pinhole along optical axis {[}m{]}.

\item {} 
\sphinxAtStartPar
\sphinxstyleliteralstrong{\sphinxupquote{pinh\_nz}} (\sphinxstyleliteralemphasis{\sphinxupquote{int}}) \textendash{} Number of points in pinhole horizontally.

\item {} 
\sphinxAtStartPar
\sphinxstyleliteralstrong{\sphinxupquote{pinh\_ny}} (\sphinxstyleliteralemphasis{\sphinxupquote{int}}) \textendash{} Number of points in pinhole vertically.

\end{itemize}

\end{description}\end{quote}

\end{fulllineitems}


\end{fulllineitems}



\subsection{Preprocess Files}
\label{\detokenize{API:preprocess-files}}\index{PreprocessWaveFiles (class in wavepy.preprocess\_wave\_files)@\spxentry{PreprocessWaveFiles}\spxextra{class in wavepy.preprocess\_wave\_files}}

\begin{fulllineitems}
\phantomsection\label{\detokenize{API:wavepy.preprocess_wave_files.PreprocessWaveFiles}}
\pysigstartsignatures
\pysiglinewithargsret{\sphinxbfcode{\sphinxupquote{class\DUrole{w,w}{  }}}\sphinxcode{\sphinxupquote{wavepy.preprocess\_wave\_files.}}\sphinxbfcode{\sphinxupquote{PreprocessWaveFiles}}}{\sphinxparam{\DUrole{n,n}{wave\_paras}\DUrole{p,p}{:}\DUrole{w,w}{  }\DUrole{n,n}{{\hyperref[\detokenize{API:wavepy.standard_parameters.StdParameters}]{\sphinxcrossref{StdParameters}}}}}}{}
\pysigstopsignatures
\sphinxAtStartPar
\_summary\_
\begin{quote}\begin{description}
\sphinxlineitem{Parameters}
\sphinxAtStartPar
\sphinxstyleliteralstrong{\sphinxupquote{wave\_paras}} ({\hyperref[\detokenize{API:wavepy.standard_parameters.StdParameters}]{\sphinxcrossref{\sphinxstyleliteralemphasis{\sphinxupquote{StdParameters}}}}}) \textendash{} Standard Parameters used for the simulation

\end{description}\end{quote}
\index{create\_wave\_input() (wavepy.preprocess\_wave\_files.PreprocessWaveFiles method)@\spxentry{create\_wave\_input()}\spxextra{wavepy.preprocess\_wave\_files.PreprocessWaveFiles method}}

\begin{fulllineitems}
\phantomsection\label{\detokenize{API:wavepy.preprocess_wave_files.PreprocessWaveFiles.create_wave_input}}
\pysigstartsignatures
\pysiglinewithargsret{\sphinxbfcode{\sphinxupquote{create\_wave\_input}}}{}{}
\pysigstopsignatures
\sphinxAtStartPar
Creates all the files needed as input fro Wave.

\sphinxAtStartPar
Loads the input file set in wave\_paras, updates properties
based on other wave\_paras properties,and copies the
resulting file to the WAVE program folder.

\end{fulllineitems}

\index{prepare\_b\_files\_for\_wave() (wavepy.preprocess\_wave\_files.PreprocessWaveFiles method)@\spxentry{prepare\_b\_files\_for\_wave()}\spxextra{wavepy.preprocess\_wave\_files.PreprocessWaveFiles method}}

\begin{fulllineitems}
\phantomsection\label{\detokenize{API:wavepy.preprocess_wave_files.PreprocessWaveFiles.prepare_b_files_for_wave}}
\pysigstartsignatures
\pysiglinewithargsret{\sphinxbfcode{\sphinxupquote{prepare\_b\_files\_for\_wave}}}{}{}
\pysigstopsignatures
\sphinxAtStartPar
Prepare the files for WAVE depending on the b type.

\sphinxAtStartPar
Deoending on which b\_type, copies and
formats the b\sphinxhyphen{}field files needed

\end{fulllineitems}


\end{fulllineitems}



\subsection{Postprocess Files}
\label{\detokenize{API:postprocess-files}}\index{PostprocessWaveFiles (class in wavepy.postprocess\_wave\_files)@\spxentry{PostprocessWaveFiles}\spxextra{class in wavepy.postprocess\_wave\_files}}

\begin{fulllineitems}
\phantomsection\label{\detokenize{API:wavepy.postprocess_wave_files.PostprocessWaveFiles}}
\pysigstartsignatures
\pysiglinewithargsret{\sphinxbfcode{\sphinxupquote{class\DUrole{w,w}{  }}}\sphinxcode{\sphinxupquote{wavepy.postprocess\_wave\_files.}}\sphinxbfcode{\sphinxupquote{PostprocessWaveFiles}}}{\sphinxparam{\DUrole{n,n}{wave\_paras}\DUrole{p,p}{:}\DUrole{w,w}{  }\DUrole{n,n}{{\hyperref[\detokenize{API:wavepy.standard_parameters.StdParameters}]{\sphinxcrossref{StdParameters}}}}}}{}
\pysigstopsignatures
\sphinxAtStartPar
A class for postprocessing WAVE files.
\begin{quote}\begin{description}
\sphinxlineitem{Parameters}
\sphinxAtStartPar
\sphinxstyleliteralstrong{\sphinxupquote{wave\_paras}} ({\hyperref[\detokenize{API:wavepy.standard_parameters.StdParameters}]{\sphinxcrossref{\sphinxstyleliteralemphasis{\sphinxupquote{StdParameters}}}}}) \textendash{} An instance of the StdParameters class.

\end{description}\end{quote}
\index{cleanup() (wavepy.postprocess\_wave\_files.PostprocessWaveFiles method)@\spxentry{cleanup()}\spxextra{wavepy.postprocess\_wave\_files.PostprocessWaveFiles method}}

\begin{fulllineitems}
\phantomsection\label{\detokenize{API:wavepy.postprocess_wave_files.PostprocessWaveFiles.cleanup}}
\pysigstartsignatures
\pysiglinewithargsret{\sphinxbfcode{\sphinxupquote{cleanup}}}{\sphinxparam{\DUrole{n,n}{wave\_folder}}}{}
\pysigstopsignatures
\sphinxAtStartPar
Cleans up the WAVE run by removing the ‘WAVE.mhb’ file if it exists in the specified folder.
\begin{quote}\begin{description}
\sphinxlineitem{Parameters}
\sphinxAtStartPar
\sphinxstyleliteralstrong{\sphinxupquote{wave\_folder}} (\sphinxstyleliteralemphasis{\sphinxupquote{str}}) \textendash{} The folder containing the WAVE run files.

\end{description}\end{quote}

\end{fulllineitems}

\index{edit\_wave\_results() (wavepy.postprocess\_wave\_files.PostprocessWaveFiles method)@\spxentry{edit\_wave\_results()}\spxextra{wavepy.postprocess\_wave\_files.PostprocessWaveFiles method}}

\begin{fulllineitems}
\phantomsection\label{\detokenize{API:wavepy.postprocess_wave_files.PostprocessWaveFiles.edit_wave_results}}
\pysigstartsignatures
\pysiglinewithargsret{\sphinxbfcode{\sphinxupquote{edit\_wave\_results}}}{}{}
\pysigstopsignatures
\sphinxAtStartPar
Cleans the wave\sphinxhyphen{}stage folder and copies the desired files to their location,
deletes non\sphinxhyphen{}desired files, and zips the results based on the wave\_res\_copy\_behaviour setting.

\end{fulllineitems}

\index{extract\_summary() (wavepy.postprocess\_wave\_files.PostprocessWaveFiles method)@\spxentry{extract\_summary()}\spxextra{wavepy.postprocess\_wave\_files.PostprocessWaveFiles method}}

\begin{fulllineitems}
\phantomsection\label{\detokenize{API:wavepy.postprocess_wave_files.PostprocessWaveFiles.extract_summary}}
\pysigstartsignatures
\pysiglinewithargsret{\sphinxbfcode{\sphinxupquote{extract\_summary}}}{\sphinxparam{\DUrole{n,n}{folder}}}{}
\pysigstopsignatures
\sphinxAtStartPar
Extracts summary information from a WAVE run’s files in the specified folder
and stores the results in the file self.wave\_paras.res\_summary\_file within the folder.
\begin{quote}\begin{description}
\sphinxlineitem{Parameters}
\sphinxAtStartPar
\sphinxstyleliteralstrong{\sphinxupquote{folder}} (\sphinxstyleliteralemphasis{\sphinxupquote{str}}) \textendash{} The folder containing the WAVE run files.

\end{description}\end{quote}

\end{fulllineitems}


\end{fulllineitems}



\subsection{Extract and Plot}
\label{\detokenize{API:extract-and-plot}}\index{ExtractAndPlot (class in wavepy.extract\_plot\_results)@\spxentry{ExtractAndPlot}\spxextra{class in wavepy.extract\_plot\_results}}

\begin{fulllineitems}
\phantomsection\label{\detokenize{API:wavepy.extract_plot_results.ExtractAndPlot}}
\pysigstartsignatures
\pysiglinewithargsret{\sphinxbfcode{\sphinxupquote{class\DUrole{w,w}{  }}}\sphinxcode{\sphinxupquote{wavepy.extract\_plot\_results.}}\sphinxbfcode{\sphinxupquote{ExtractAndPlot}}}{\sphinxparam{\DUrole{n,n}{wave\_paras}}}{}
\pysigstopsignatures\index{extract\_wave\_results() (wavepy.extract\_plot\_results.ExtractAndPlot method)@\spxentry{extract\_wave\_results()}\spxextra{wavepy.extract\_plot\_results.ExtractAndPlot method}}

\begin{fulllineitems}
\phantomsection\label{\detokenize{API:wavepy.extract_plot_results.ExtractAndPlot.extract_wave_results}}
\pysigstartsignatures
\pysiglinewithargsret{\sphinxbfcode{\sphinxupquote{extract\_wave\_results}}}{\sphinxparam{\DUrole{n,n}{results}\DUrole{p,p}{:}\DUrole{w,w}{  }\DUrole{n,n}{list}}, \sphinxparam{\DUrole{n,n}{plot}\DUrole{p,p}{:}\DUrole{w,w}{  }\DUrole{n,n}{bool}\DUrole{w,w}{  }\DUrole{o,o}{=}\DUrole{w,w}{  }\DUrole{default_value}{False}}, \sphinxparam{\DUrole{n,n}{en\_range}\DUrole{p,p}{:}\DUrole{w,w}{  }\DUrole{n,n}{list}\DUrole{w,w}{  }\DUrole{o,o}{=}\DUrole{w,w}{  }\DUrole{default_value}{{[}{]}}}, \sphinxparam{\DUrole{n,n}{xlim}\DUrole{p,p}{:}\DUrole{w,w}{  }\DUrole{n,n}{list}\DUrole{w,w}{  }\DUrole{o,o}{=}\DUrole{w,w}{  }\DUrole{default_value}{{[}{]}}}, \sphinxparam{\DUrole{n,n}{ylim}\DUrole{p,p}{:}\DUrole{w,w}{  }\DUrole{n,n}{list}\DUrole{w,w}{  }\DUrole{o,o}{=}\DUrole{w,w}{  }\DUrole{default_value}{{[}{]}}}, \sphinxparam{\DUrole{n,n}{save\_data\_w\_pics}\DUrole{p,p}{:}\DUrole{w,w}{  }\DUrole{n,n}{bool}\DUrole{w,w}{  }\DUrole{o,o}{=}\DUrole{w,w}{  }\DUrole{default_value}{False}}}{}
\pysigstopsignatures
\sphinxAtStartPar
Extracts the desired results and returns a DataFrame with them.
\begin{quote}\begin{description}
\sphinxlineitem{Parameters}\begin{itemize}
\item {} 
\sphinxAtStartPar
\sphinxstyleliteralstrong{\sphinxupquote{results}} (\sphinxstyleliteralemphasis{\sphinxupquote{list}}) \textendash{} A list containing the desired result types. Valid options are:
\sphinxhyphen{} ‘traj\_magn’: Extracts Trajectory and magnetic field data.
\sphinxhyphen{} ‘power\_distr’: Power distribution {[}W/mm\textasciicircum{}2{]}.
\sphinxhyphen{} ‘flux’: Flux.
\sphinxhyphen{} ‘flux\_dens’: Flux Density.
\sphinxhyphen{} ‘flux\_dens\_distr’: Flux Density Distribution.
\sphinxhyphen{} ‘power\_dens\_distr’: Power density distribution.
\sphinxhyphen{} ‘brill’: Brilliance.

\item {} 
\sphinxAtStartPar
\sphinxstyleliteralstrong{\sphinxupquote{plot}} (\sphinxstyleliteralemphasis{\sphinxupquote{bool}}) \textendash{} Determines if loaded data is plotted or not.

\item {} 
\sphinxAtStartPar
\sphinxstyleliteralstrong{\sphinxupquote{en\_range}} (\sphinxstyleliteralemphasis{\sphinxupquote{list}}) \textendash{} For distribution, gives the energy range over which the data is integrated.

\item {} 
\sphinxAtStartPar
\sphinxstyleliteralstrong{\sphinxupquote{xlim}} (\sphinxstyleliteralemphasis{\sphinxupquote{list}}) \textendash{} Sets the limits of the distribution plots in the horizontal direction.

\item {} 
\sphinxAtStartPar
\sphinxstyleliteralstrong{\sphinxupquote{ylim}} (\sphinxstyleliteralemphasis{\sphinxupquote{list}}) \textendash{} Sets the limits of the distribution plots in the vertical direction. For 2D plots, only xlim is considered if given.

\item {} 
\sphinxAtStartPar
\sphinxstyleliteralstrong{\sphinxupquote{save\_data\_w\_pics}} (\sphinxstyleliteralemphasis{\sphinxupquote{bool}}) \textendash{} If True, saves the data for each plot alongside the picture.

\end{itemize}

\sphinxlineitem{Returns}
\sphinxAtStartPar
A list of the loaded DataFrames.

\sphinxlineitem{Return type}
\sphinxAtStartPar
list

\end{description}\end{quote}

\end{fulllineitems}

\index{load\_plot\_stokes\_distrib() (wavepy.extract\_plot\_results.ExtractAndPlot method)@\spxentry{load\_plot\_stokes\_distrib()}\spxextra{wavepy.extract\_plot\_results.ExtractAndPlot method}}

\begin{fulllineitems}
\phantomsection\label{\detokenize{API:wavepy.extract_plot_results.ExtractAndPlot.load_plot_stokes_distrib}}
\pysigstartsignatures
\pysiglinewithargsret{\sphinxbfcode{\sphinxupquote{load\_plot\_stokes\_distrib}}}{\sphinxparam{\DUrole{n,n}{folder}}, \sphinxparam{\DUrole{n,n}{en\_range}\DUrole{o,o}{=}\DUrole{default_value}{{[}{]}}}, \sphinxparam{\DUrole{n,n}{xlim}\DUrole{o,o}{=}\DUrole{default_value}{{[}{]}}}, \sphinxparam{\DUrole{n,n}{ylim}\DUrole{o,o}{=}\DUrole{default_value}{{[}{]}}}, \sphinxparam{\DUrole{n,n}{plot}\DUrole{o,o}{=}\DUrole{default_value}{True}}, \sphinxparam{\DUrole{n,n}{save\_folder}\DUrole{o,o}{=}\DUrole{default_value}{\textquotesingle{}\textquotesingle{}}}, \sphinxparam{\DUrole{n,n}{save\_data\_w\_pics}\DUrole{o,o}{=}\DUrole{default_value}{False}}, \sphinxparam{\DUrole{n,n}{power\_distr}\DUrole{o,o}{=}\DUrole{default_value}{False}}}{}
\pysigstopsignatures
\sphinxAtStartPar
loads stokes (flux\sphinxhyphen{}density) data from a folder, integrates the flux\sphinxhyphen{}density distribution over the en\_range
and cuts the plot horizontally and vertically to xlim and ylim, plots if plot = True and saves pic to save\_folder
returns the loaded and integrated data object
If power\_distr is True, then not the flux density but power density in the given energy
range is calculated (multiplying all fluxes by the appropriate energies)

\end{fulllineitems}


\end{fulllineitems}




\renewcommand{\indexname}{Index}
\printindex
\end{document}